%%%%%%%%%%%%%%%%%%%%%%%%%%%%%%%%%%%%%%%%%
% Beamer Presentation
% LaTeX Template
% Version 1.0 (10/11/12)
%
% This template has been downloaded from:
% http://www.LaTeXTemplates.com
%
% License:
% CC BY-NC-SA 3.0 (http://creativecommons.org/licenses/by-nc-sa/3.0/)
%
%%%%%%%%%%%%%%%%%%%%%%%%%%%%%%%%%%%%%%%%%

%------------------------------------------------
%	PACKAGES AND THEMES
%------------------------------------------------

\documentclass{beamer}

\mode<presentation> {

% The Beamer class comes with a number of default slide themes
% which change the colors and layouts of slides. Below this is a list
% of all the themes, uncomment each in turn to see what they look like.

%\usetheme{default}
%\usetheme{AnnArbor}
%\usetheme{Antibes}
%\usetheme{Bergen}
%\usetheme{Berkeley}
%\usetheme{Berlin}
%\usetheme{Boadilla}
%\usetheme{CambridgeUS}
%\usetheme{Copenhagen}
%\usetheme{Darmstadt}
%\usetheme{Dresden}
%\usetheme{Frankfurt}
%\usetheme{Goettingen}
%\usetheme{Hannover}
%\usetheme{Ilmenau}
%\usetheme{JuanLesPins}
%\usetheme{Luebeck}
\usetheme{Madrid}
%\usetheme{Malmoe}
%\usetheme{Marburg}
%\usetheme{Montpellier}
%\usetheme{PaloAlto}
%\usetheme{Pittsburgh}
%\usetheme{Rochester}
%\usetheme{Singapore}
%\usetheme{Szeged}
%\usetheme{Warsaw}

% As well as themes, the Beamer class has a number of color themes
% for any slide theme. Uncomment each of these in turn to see how it
% changes the colors of your current slide theme.

%\usecolortheme{albatross}
%\usecolortheme{beaver}
%\usecolortheme{beetle}
%\usecolortheme{crane}
%\usecolortheme{dolphin}
%\usecolortheme{dove}
%\usecolortheme{fly}
%\usecolortheme{lily}
%\usecolortheme{orchid}
%\usecolortheme{rose}
%\usecolortheme{seagull}
%\usecolortheme{seahorse}
%\usecolortheme{whale}
%\usecolortheme{wolverine}

%\setbeamertemplate{footline} % To remove the footer line in all slides uncomment this line
%\setbeamertemplate{footline}[page number] % To replace the footer line in all slides with a simple slide count uncomment this line

\setbeamertemplate{navigation symbols}{} % To remove the navigation symbols from the bottom of all slides uncomment this line
}

\usepackage{graphicx} % Allows including images
\usepackage{booktabs} % Allows the use of \toprule, \midrule and \bottomrule in tables

%------------------------------------------------
%	TITLE PAGE
%------------------------------------------------

\title[Capture The Flag]{Découverte de la cybersécurité via les Capture The Flag (CTF)} % The short title appears at the bottom of every slide, the full title is only on the title page

\author{Ugo Proietti et François Vion} % Your name
\institute[UMONS] % Your institution as it will appear on the bottom of every slide, may be shorthand to save space
{
Universtité de Mons \\ % Your institution for the title page
\medskip
}
\date{\today} % Date, can be changed to a custom date

\begin{document}

\begin{frame}
\titlepage % Print the title page as the first slide
\end{frame}

\begin{frame}
\frametitle{} % Table of contents slide, comment this block out to remove it
\tableofcontents % Throughout your presentation, if you choose to use \section{} and \subsection{} commands, these will automatically be printed on this slide as an overview of your presentation
\end{frame}

%------------------------------------------------
%	PRESENTATION SLIDES
%------------------------------------------------

%------------------------------------------------
\section{Qu'est-ce qu'un CTF ?} % Sections can be created in order to organize your presentation into discrete blocks, all sections and subsections are automatically printed in the table of contents as an overview of the talk
%------------------------------------------------

\begin{frame}
\frametitle{Principe}
Les Capture The Flag sont des compétitions durant lesquelles vous allez devoir trouver des flags. Ils suivent un format standard en fonction de l'évènement \{CSC0123456789\}. \\
Chaque flag rapporte un certain nombre de points. \\~\\

Il existe un grand nombre de catégories ayant chacune leurs caractéristiques, nous allons vous présenter les plus connues.
\end{frame}

%------------------------------------------------

\begin{frame}
\frametitle{Catégories}

\begin{itemize}
    \item Stéganographie
    \item Cryptographie
    \item Web
    \item Réseau
    \item Forensic
    \item Reverse Engineering
\end{itemize}

\end{frame}

%------------------------------------------------

\begin{frame}
\frametitle{Stéganographie}

La Stéganographie est l’art de la dissimulation. Il s’agit en général de cacher une information là où on ne s’y attend pas. \\~\\

\begin{block}{Exemples}
    Cacher un message dans un fichier audio \\
    Message écrit en très petit sur une image \\
    Cacher une image dans une autre image
\end{block}

\begin{block}{Comptétences utiles}
    Observation \\
    Utilisation de scripts \\
    Réflexion
\end{block}

\begin{block}{Difficulté}
    $1/5$
\end{block}

\end{frame}

%------------------------------------------------

\begin{frame}
\frametitle{Cryptographie}

Contrairement à la stéganographie, la cryptograhpie n'essaye pas de cacher un message. Elle se contente de le rendre illisible par toute personne n'ayant pas la clé pour le déchiffrer.

\begin{block}{Exemples}
    Cryptage d'un message 
\end{block}

\begin{block}{Compétences}
    Connaissance de divers formats d'encodage \\
    Connaissances mathématiques pour craquer un code
\end{block}

\begin{block}{Difficulté}
    $2/5$
\end{block}

\end{frame}

%------------------------------------------------

\begin{frame}
\frametitle{Web}

Un site web peut cacher un mot de passe, un fichier ou une faille permettant de s'y connecter sans autorisations.

\begin{block}{Exemples}
\end{block}

\begin{block}{Compétences}
\end{block}

\begin{block}{Difficulté}
\end{block}

\end{frame}

%------------------------------------------------

\begin{frame}
\frametitle{Réseau}

Surveiller un réseau peut révéler des informations concernant les clients connectés et les données qui y sont échangées. \\
On peut imaginer récupérer des infentifiants de connexion en interceptant ce qui transite sur le réseau. 

\begin{block}{Exemples}
\end{block}

\begin{block}{Compétences}
\end{block}

\begin{block}{Difficulté}
\end{block}

\end{frame}

%------------------------------------------------

\begin{frame}
\frametitle{Forensic}

Une machine peut contenir des informations critiques qui sont récupérables en exploitant la mémoire, le stockage ou des logs générés par le système d'exploitation. \\
La mémoire peut contenir des mots de passe utilisés par l'OS ou des tokens de connection utilisés par le navigateur web.

\begin{block}{Exemples}
\end{block}

\begin{block}{Compétences}
\end{block}

\begin{block}{Difficulté}
\end{block}

\end{frame}

%------------------------------------------------

\begin{frame}
\frametitle{Reverse Engineering}

Un fichier executable peut être analysé pour comprendre son fonctionnement et en tirer des informations.

\begin{block}{Exemples}
\end{block}

\begin{block}{Compétences}
\end{block}

\begin{block}{Difficulté}
\end{block}

\end{frame}

%------------------------------------------------

\begin{frame}
\frametitle{Blocks of Highlighted Text}
\begin{block}{Block 1}
Lorem ipsum dolor sit amet, consectetur adipiscing elit. Integer lectus nisl, ultricies in feugiat rutrum, porttitor sit amet augue. Aliquam ut tortor mauris. Sed volutpat ante purus, quis accumsan dolor.
\end{block}

\begin{block}{Block 2}
Pellentesque sed tellus purus. Class aptent taciti sociosqu ad litora torquent per conubia nostra, per inceptos himenaeos. Vestibulum quis magna at risus dictum tempor eu vitae velit.
\end{block}

\begin{block}{Block 3}
Suspendisse tincidunt sagittis gravida. Curabitur condimentum, enim sed venenatis rutrum, ipsum neque consectetur orci, sed blandit justo nisi ac lacus.
\end{block}
\end{frame}

%------------------------------------------------

\begin{frame}
\frametitle{Multiple Columns}
\begin{columns}[c] % The "c" option specifies centered vertical alignment while the "t" option is used for top vertical alignment

\column{.45\textwidth} % Left column and width
\textbf{Heading}
\begin{enumerate}
\item Statement
\item Explanation
\item Example
\end{enumerate}

\column{.5\textwidth} % Right column and width
Lorem ipsum dolor sit amet, consectetur adipiscing elit. Integer lectus nisl, ultricies in feugiat rutrum, porttitor sit amet augue. Aliquam ut tortor mauris. Sed volutpat ante purus, quis accumsan dolor.

\end{columns}
\end{frame}

%------------------------------------------------
\section{Aperçu du fonctionnement}
%------------------------------------------------

\begin{frame}
\frametitle{Table}
\begin{table}
\begin{tabular}{l l l}
\toprule
\textbf{Treatments} & \textbf{Response 1} & \textbf{Response 2}\\
\midrule
Treatment 1 & 0.0003262 & 0.562 \\
Treatment 2 & 0.0015681 & 0.910 \\
Treatment 3 & 0.0009271 & 0.296 \\
\bottomrule
\end{tabular}
\caption{Table caption}
\end{table}
\end{frame}

%------------------------------------------------

\begin{frame}
\frametitle{Theorem}
\begin{theorem}[Mass--energy equivalence]
$E = mc^2$
\end{theorem}
\end{frame}

%------------------------------------------------

\begin{frame}[fragile] % Need to use the fragile option when verbatim is used in the slide
\frametitle{Verbatim}
\begin{example}[Theorem Slide Code]
\begin{verbatim}
\begin{frame}
\frametitle{Theorem}
\begin{theorem}[Mass--energy equivalence]
$E = mc^2$
\end{theorem}
\end{frame}\end{verbatim}
\end{example}
\end{frame}

%------------------------------------------------

\begin{frame}
\frametitle{Figure}
Uncomment the code on this slide to include your own image from the same directory as the template .TeX file.
%\begin{figure}
%\includegraphics[width=0.8\linewidth]{test}
%\end{figure}
\end{frame}

%------------------------------------------------

\begin{frame}[fragile] % Need to use the fragile option when verbatim is used in the slide
\frametitle{Citation}
An example of the \verb|\cite| command to cite within the presentation:\\~

This statement requires citation \cite{p1}.
\end{frame}

%------------------------------------------------

\section{Mise en pratique}
\begin{frame}
\frametitle{Mise en pratique}

    C'est votre tour ! Créez un comte sur root-me.org et essayez les exercices faciles

\end{frame}

\end{document} 
